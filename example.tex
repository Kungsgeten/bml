\documentclass[a4paper]{article}
\usepackage[T1]{fontenc}
\usepackage[utf8]{inputenc}
\usepackage{newcent}
\usepackage{helvet}
\usepackage{graphicx}
\usepackage[pdftex, pdfborder={0 0 0}]{hyperref}
\frenchspacing

\usepackage{txfonts} % For \varheartsuit and \vardiamondsuit
\usepackage[usenames,dvipsnames]{color} % dvipsnames necessary to made PDFLaTeX work.
% \usepackage{latexsym} % \Box
% \usepackage{pbox}


% suits

%%% Black and white

% \renewcommand{\c}{$\clubsuit$}
% \renewcommand{\d}{$\diamondsuit$}
% \newcommand{\h}{$\heartsuit$}
% \newcommand{\s}{$\spadesuit32$}

%%% Colors

\renewcommand{\c}{\textcolor{OliveGreen}{$\clubsuit$}}
\renewcommand{\d}{\textcolor{RedOrange}{$\vardiamondsuit$}}
\newcommand{\h}{\textcolor{Red}{$\varheartsuit${}}}
\newcommand{\s}{\textcolor{Blue}{$\spadesuit${}}}

%suits for pdf-friendly titles
\newcommand{\pdfc}{\texorpdfstring{\c{}}{C}}
\newcommand{\pdfd}{\texorpdfstring{\d{}}{D}}
\newcommand{\pdfh}{\texorpdfstring{\h{}}{H}}
\newcommand{\pdfs}{\texorpdfstring{\s{}}{S}}

\newenvironment{bidtable}
{\begin{tabbing}

xxxxxxxxx\=xxxxxxxxx\=xxxxxxxxx\=xxxxxxxxx\=xxxxxxxxx\=xxxxxxxxx\=\kill}
{\end{tabbing} }%

% writing hands
\newcommand{\cards}[1]{\textsf{#1}}
\newcommand{\spades}[1]{\s\cards{#1}}
\newcommand{\hearts}[1]{\h\cards{#1}}
\newcommand{\diamonds}[1]{\d\cards{#1}}
\newcommand{\clubs}[1]{\c\cards{#1}}
\newcommand{\void}{--}
\newcommand{\hand}[4]{\spades{#1}\ \hearts{#2}\ \diamonds{#3} \clubs{#4}}
\newcommand{\vhand}[4]{\spades{#1}\\\hearts{#2}\\\diamonds{#3}\\\clubs{#4}}

\newcommand\onesuit[4]%
{%
  \begin{center}%
    \begin{tabular}{>{\hfill}p{3cm}cp{3cm}}
                & \cards{#2} \\%
      \cards{#1}& $\Box$    & \cards{#3} \\%
                & \cards{#4} %
    \end{tabular}
  \end{center}%
}

\newcommand{\dealdiagram}[4]
{
  \begin{tabular}{lll}
    & \pbox{20cm}{#2} \\
    \pbox{20cm}{#1} & $\Box$ & \pbox{20cm}{#3} \\
    & \pbox{20cm}{#4} \\
  \end{tabular}
}

\title{BML 5542}
\author{Erik Sjöstrand}
\begin{document}
\maketitle
\tableofcontents

\section{Introduction}

Welcome to BML! This is a normal paragraph, and above we can see
the TITLE, the AUTHOR and the DESCRIPTION of the file. TITLE is
the name of the system and DESCRIPTION is a short summary of how
the system works. AUTHOR is self explanatory. Introduction above,
headed by an asterisk, sets a section at the first level (the second
level would have two asterisks etc).

The system presented in this example file is meant to showcase many
of the current features in BML. Let's start with the basic opening
structure of the system:

\begin{bidtable}
1\c \> 2+\c. Natural / 11--13 bal / 17--19 bal\\
1\d \> 4+ suit, unbalanced\\
1M \> 5+ suit\\
1NT \> 14--16\\
2\c \> 20--21 bal / Any game force\\
2\d \> 6+\h\ or 6+\s, 5--9 hcp\\
2\h\s \> 6+ suit, 10--13 hcp\\
2NT \> 22--24\\
3X \> Preemptive\\
3NT \> Gambling
\end{bidtable}

The above is an example of a bidding table; the reason why BML is
more suited for bridge system notes than other markup languages. You
start by writing the bid, then a number of whitespaces, and then the
description of the bid. Simple! C is for clubs, D for diamonds, H
for hearts, S for spades and N for no trump. There's also some
special cases which you could use, above we use 1M (1H and 1S), 2HS
(2H and 2S) and 3X (3C, 3D, 3H and 3S). We'll see more of these
later.

\section{The 1\pdfc\ opening}

You might have noticed the \c\ in the title of this section? This
will be replaced by a club suit symbol when exported. The same is
true for \d, \h\ and \s\ (but these will be converted to diamonds,
hearts and spades, ofcourse).

In this example we use transfer responses to the 1\c\ opening:

\begin{bidtable}
1\c---\\
1red \> Transfer. 4+ major, 0+ hcp\\
1\s \> INV+ with 5+\d\ / Negative NT\\
1NT \> Game forcing, 5+\c\ or balanced\\
2\c \> 5+\c, 5--9 hcp\\
2X \> 6+ suit, 4--8 hcp\\
v2NT \> Invitational
\end{bidtable}

By writing 1C--- we define that the following bids should be
continuations to the sequence 1C. We could write 1C- or 1C-- too,
the number of dashes only matters to the way the output looks. Also
note the 1red response, this defines both 1D and 1H.

\subsection{After a transfer}

This section has two asterisks, meaning it will be at level two
(so its a subsection). You might also have noticed that the
paragraphs, the sections and the bidtables are separated by a
blank line? This is important in BML, as the blankline are used to
separate elements.

\begin{bidtable}
1\c-1\d;\\
1\h \> Minimum with 2--3\h\+\\
1\s \> 4+\h, 4\s, at most invitational\\
1NT \> Sign off\\
2\c \> Puppet to 2\d\+\\
2\d \> Forced\+\\
2\h \> Mildly invitational with 5\h\\
2\s \> Invitational, 5+\h\ and 4\s\\
2NT \> Strongly invitational with 5\h\\
3m \> Invitational with 4\h\ and 5+ minor\\
3\h \> 6\h, about 11--12 hcp\-\-\\
2\d \> Artificial game force\\
2\h \> 6+\h, about 9--10 hcp\-\\
1\s \> 5+\c, 4+\s, unlimited\\
1NT \> 17--19 bal, 2--3\h\\
2\c \> 5+\c, unbal, 0--2\h, 0--3\s\\
2\d \> Reverse\\
2\h \> Minimum, 4\h\\
2\s \> 16+ hcp, 5+\c\ and 4+\h\+\\
3\d \> Retransfer\+\\
3\h\+\\
3\s \> Cue bid, slam interest\\
4\c\d \> Cue bid, slam interest\\
4\h \> To play\-\-\\
3\h \> Invitational\\
3\s \> Splinter\\
4\c\d \> Splinter\\
4\h \> To play\-\\
2NT \> 16+ hcp, 6+\c. 18+ if 3\h\+\\
3\c \> Suggestion to play\\
3\d \> Relay\+\\
3\h \> 3\h, 18+ hcp\-\\
3\h \> Game forcing with 6+\h\-\\
3\c \> 15--17 hcp, 6+\c\ and 3\h\+\\
3\d \> Retransfer\\
3\h \> Invitational\-\\
3\d \> 17--19 bal, 4\h\+\\
3\h \> To play\-\\
3\h \> 13--15 hcp, good hand, 5+\c\ and 4\h\+\\
3NT \> Asking for singleton\-
\end{bidtable}

This bidding table shows a couple of new features. The most
prominent is the ability to add continuations directly in the
table, by using whitespaces. We also see another example of
appending bids to an existing sequence, by using 1C-1D; in the
beginning. There's also the use of 3m, meaning both 3C and 3D.

\section{Defense to 1NT}

Defining bidding when both sides bid is a little bit more tricky,
since you have to write all the bids (even passes). The opponents'
bid are indicated by encapsulating them in parantheses:

\begin{bidtable}
(1NT)---\\
D \> Strength, ca 15+\\
2\c \> At least 5-4 majors\+\\
(D)\+\\
P \> 5+\c, suggestion to play\\
R \> Asking for better/longer major\\
2\d \> 5+\d, suggestion to play\-\\
(P)\+\\
2\d \> Asking for better/longer major\-\-\\
2\d \> A weak major or a strong minor\+\\
(P)\+\\
2\h \> Pass/correct\\
2\s \> Pass/correct\\
2NT \> Asking\-\-\\
2\h\s \> Constructive\\
2NT \> 5-5 minors\\
3X \> Preemptive
\end{bidtable}

P is used for Pass, D for Double and R for Redouble. Note that the
above is only for a direct overcall over 1NT. To define the above
also when balancing, you would have to write:

\begin{bidtable}
(1NT)-P-(P)---\\
D \> Strength, ca 15+\\
2\c \> At least 5-4 majors\+\\
(D)\+\\
P \> 5+\c, suggestion to play\\
R \> Asking for better/longer major\\
2\d \> 5+\d, suggestion to play\-\\
(P)\+\\
2\d \> Asking for better/longer major\-\-\\
2\d \> A weak major or a strong minor\+\\
(P)\+\\
2\h \> Pass/correct\\
2\s \> Pass/correct\\
2NT \> Asking\-\-\\
2\h\s \> Constructive\\
2NT \> 5-5 minors\\
3X \> Preemptive
\end{bidtable}

You might also do this when the opponents interfere:

\begin{bidtable}
1\c-(1\d)---\\
D \> 4+\h\\
1\h \> 4+\s\\
1\s \> INV+ with 5+\d\ / Negative NT\\
1NT \> Game forcing, 5+\c\ or balanced\\
2\c \> 5+\c, 5--9 hcp\\
2X \> 6+ suit, 4--8 hcp\\
2NT \> Invitational
\end{bidtable}

\section{Lists}

At last I'd like to show you how to make lists in BML. It is pretty
simple:

\begin{itemize}
\item Here's a list!

\item With a couple of

\item Items in it

\end{itemize}

You could also make ordered lists:

\begin{enumerate}
\item This is ordered

\item Just add numbers

\item To each item


\end{enumerate}

\end{document}
